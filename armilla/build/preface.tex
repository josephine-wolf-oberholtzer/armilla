\documentclass[10pt]{article}
\usepackage{enumitem}
\usepackage[utf8]{inputenc}
\usepackage[papersize={11in, 17in}]{geometry}
\usepackage[absolute]{textpos}
\TPGrid[0.5in, 0.25in]{23}{24}
\usepackage{palatino}
\parindent=0pt
\parskip=12pt
\usepackage{nopageno}

\begin{document}

\begin{textblock}{23}(0,1)
\center\huge PREFACE
\end{textblock}

\begin{textblock}{11}(0, 3)

\section{}

\textit{Whether Armilla is like this because it is unfinished or because it has 
been demolished, whether the cause is some enchantment or only a whim, I do not
know. The fact remains that it has no walls, no ceilings, no floors: it has
nothing that makes it seem a city except the water pipes that rise vertically
where the houses should be and spread out horizontally where the floors should
be: a forest of pipes that end in taps, shouwers, spouts, overflows. Against
the sky a lavabo's white stands out, or a bathtub, or some other porcelain,
like late fruit still hanging from the boughs. You would think that the
plumbers had finished their job and gone away before the bricklayers arrived;
or else their hydraulic systems, indestructable, had survived a catastrophe, an
earthquake, or the corrosion of termites. }

\textit{Abandoned before or after it was inhabited, Armilla cannot be called
deserted. At any hour, raising your eyes among the pipes, you are likely to
glimpse a young woman, or many young women, slender, not tall of stature,
luxuriating in the bathtubs or arching their backs under the showers suspended
in the void, washing or drying or perfuming themselves, or combing their long
hair at a mirror. In the sun, the threads of water fanning from the showers
glisten, the jets of the taps, the spurts, the splases, the sponges' suds. }

\textit{I have come to this explaination: the streams of water channeled in the
pipes of Armilla have remained in th posession of nymphs and naiads. Accustomed
to traveling along underground veins, they found it easy to enter the new
aquatic realm, to burst from multiple fountains, to find new mirrors, new
games, new ways of enjoying the water. Their invasion may have driven out the
human beings, or Armilla may have been built by humans as a votive offering to
win the favor of the nymphs, offended at the misuse of the waters. In any case,
now they seem content, these maidens: in the morning you hear them singing. }

- Italo Calvino, \emph{Invisible Cities}

\section{}

\textit{The dunes ran inland, low and grassy, for half a mile or so, and then
there were lagoons, thick with sedge and saltreeds, and beyond those, low hills
lay yellow-brown and empty out of sight. Beautiful and desolate was Selidor.
Nowhere on it was there any mark of man, his work or habitation. There were no
beasts to be seen, and the reed-filled lakes bore no flocks of gulls or wild
geese or any bird. }

- Ursula Le Guin, \emph{The Farthest Shore}

\end{textblock}

\begin{textblock}{11}(12,3)

\section{Bow contact points}

The current position along the bow as it contacts the strings is indicated
with fractions, where 0/1 indicates the frog, and 1/1 indicates the tip of
the bow. Continuous bowing is shown by lines connecting bow contact
fractions.

\section{Dynamics}

Dynamics are always in terms of effort, not effect. When bowing very
quickly and with strong dynanic, the effect should be a traditional
\emph{forte}. Likewise, when bowing slowly and with a light dynamic, the
effect should be a traditional \emph{piano}. Slow bowing with strong
dynamic should result in various colors of scratch, while fast bowing with
light pressure should give various qualities of flautando (depending of
course on where on the string the bow is contacting).

\section{String contact points}

\begin{description}[style=nextline]
    \item[D.P.]
        Dietro ponticello: behind the bridge.
    \item[M.S.P.]
        Molto sul ponticello
    \item[S.P.]
        Sul ponticello
    \item[Ord.]
        Ordinario
    \item[S.T.]
        Sul tasto
    \item[M.S.T.]
        Molto sul tasto
\end{description}

When bowing behind the bridge, the fingering staff switches to percussion
clef. The behind-the-bridge string to bow on are given by the four spaces
of the five-line-staff, with string \emph{IV} being the lowest space and
\emph{I} the highest.

\section{Other techniques}

\begin{description}[style=nextline]
    \item[Across-the-string tremoli]
        Indicated by traditional tremolo hashes on the bow tablature's
        rhythm staff
    \item[Along-the-string tremoli]
        Indicated by zigzag bow tablature glissandi. 
    \item[Thrown bow]
        Indicated by dashed bow tablature glissandi
    \item[Pizzicati]
        Indicated with a cross notehead in the tablature staff.
    \item[Accents]
        Accents in the bow tablature staff indicate a sudden staccato
        increase in bow pressure.
\end{description}

Tremoli, both across- and along-the-string, should be very tight. When the
two techniques appear simultaneously, the resulting motion is tightly
circular bowing.

\end{textblock}

\end{document}