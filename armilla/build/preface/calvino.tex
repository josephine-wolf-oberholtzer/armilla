\section{}

\textit{Whether Armilla is like this because it is unfinished or because it has 
been demolished, whether the cause is some enchantment or only a whim, I do not
know. The fact remains that it has no walls, no ceilings, no floors: it has
nothing that makes it seem a city except the water pipes that rise vertically
where the houses should be and spread out horizontally where the floors should
be: a forest of pipes that end in taps, shouwers, spouts, overflows. Against
the sky a lavabo's white stands out, or a bathtub, or some other porcelain,
like late fruit still hanging from the boughs. You would think that the
plumbers had finished their job and gone away before the bricklayers arrived;
or else their hydraulic systems, indestructable, had survived a catastrophe, an
earthquake, or the corrosion of termites. }

\textit{Abandoned before or after it was inhabited, Armilla cannot be called
deserted. At any hour, raising your eyes among the pipes, you are likely to
glimpse a young woman, or many young women, slender, not tall of stature,
luxuriating in the bathtubs or arching their backs under the showers suspended
in the void, washing or drying or perfuming themselves, or combing their long
hair at a mirror. In the sun, the threads of water fanning from the showers
glisten, the jets of the taps, the spurts, the splases, the sponges' suds. }

\textit{I have come to this explaination: the streams of water channeled in the
pipes of Armilla have remained in th posession of nymphs and naiads. Accustomed
to traveling along underground veins, they found it easy to enter the new
aquatic realm, to burst from multiple fountains, to find new mirrors, new
games, new ways of enjoying the water. Their invasion may have driven out the
human beings, or Armilla may have been built by humans as a votive offering to
win the favor of the nymphs, offended at the misuse of the waters. In any case,
now they seem content, these maidens: in the morning you hear them singing. }

- Italo Calvino, \emph{Invisible Cities}
